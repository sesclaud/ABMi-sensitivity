\chapter{Introduction} \label{ch:intro}


Transportation systems are very complex, as they involve many components and stakeholders. Among the main challenges of recent years there is that of creating a mobility infrastructure that satisfies the needs of all stakeholders, in terms of efficient movement, sustainability, safety and accessibility. The increase in population in urban centers has made it more challenging for transportation companies and public authorities to maintain high quality standards of the transportation services. In this context, transport models, digital replicas of these complex real-world systems, can be seen as a powerful tool for assessing the efficiency of transport infrastructures, and for predicting future performance in response to the dynamic needs of users. Transport models allow planners to understand the current issues in their transportation system, to identify opportunities, and to design reliable and resilient services which optimally serve passengers needs. 
In this regard, the city of Milan provides an interesting case study, with its transportation system including a total of 143 routes among metro, tram, bus and filobus, managed by Azienda Trasporti Milanesi (ATM), as well as taxi, car, bike and scooter sharing services. Every year, ATM services host around 585 million passengers, spending, on average, 43 min on public transport per day \cite{bib2}. The goal of this research is to propose a model for the demand on Milan's public transportation system, which can be used to project future demands, to test in advance the impact of interventions and changes to the network, as well as the application of polices in the mobility sector. In other words, the proposed model is a tool that attempts to evaluate the network efficiency from the emergence of issues, which in practice arise daily, such as long waiting times at the stops, overcrowded vehicles and sudden changes of the network. For simplicity, I only focus on the public transportation system and on the share of daily passengers relying on the public service.
This research is articulated around two main questions:
\begin{enumerate}
    \item Is it possible to construct a model that reproduces Milan's transportation system's infrastructure and simulates the flows of passengers making use of the service, in a way that allows evaluating how much the system is efficient in meeting users' demand?
    \item Given that it is possible to construct such a parametric model for public transportation, which are the elements to which the model is more sensible? How do model results vary in response to changes in parameter values?  
\end{enumerate}
I will show that a suitable framework to achieve the first goal of the analysis is that of Agent-based Modeling (ABM), a flexible simulation technique allowing to model a system from its constituent units. Since the information about individual trips is restricted, as Origin-destination (OD) matrices\footnote{Origin Destination Matrix contains the starting point and end point of individual trips, and it could be obtained through surveys or directly from devices owned by the transportation company, as in the ATM case. It provides a description of movements in a certain geographic area, and is used to assess the demand for transportation.} are owned by ATM, I needed to come up with a way to replicate the passenger flow. I will describe a synthesis procedure, based on publicly available aggregate statistics about the population, which will be used to create the synthetic agents to be fed to the model and simulate passengers and their travel behavior. Later, I will address the second question through a Sensitivity Analysis of the model, with respect to its parameters and "non-parametric" elements, a procedure which allows gathering insights about the dependencies and validity of the ABM's results. \\ 
Specifically, in Chapter \ref{ch:literature}, I will present an overview of the two theoretical frameworks at the basis of this research, Agent-based Modeling and Sensitivity Analysis, as to provide the readers with the concepts that are relevant for the following chapters. \\ Then, Chapter \ref{ch:model} is dedicated to a detailed explanation of the model for passengers' flow on Milan's public transportation network. Finally, in Chapter \ref{ch:sa} I will carry on a local Sensitivity Analysis of the model, following the protocol that has been recently proposed by \textcite{Borgonovo2022SensitivityAO}.

Both the simulator and the Sensitivity Analysis have been developed in Python 3.9, and the main libraries exploited are Pandas \cite{mckinney-proc-scipy-2010} and NetworkX \cite{SciPyProceedings_11}. The data and the code are available at [METTERE LINK GITHUB].