\chapter{Conclusion}

This thesis aimed at proposing a transport model for the passengers' flow on Milan's public transportation network and conducting an analysis of the sensitivity of this model to variations in its inputs. What has been presented is an Agent-based model, in which passengers represent the agents and the environment is a directed multigraph reproducing the public transportation network of the city. 
This model allows us to reproduce real-life circumstances, like changes of the network, mobility policies or temporary interventions: the experiments conducted show the emergence of realistic patterns, which is an evidence of the trustworthiness of the model.
Furthermore, I have selected three outputs that one obtains from the model evaluations --- mean waiting time per destination, mean traveling time per km, the number of edges reaching the capacity constraint for more than 30 min --- and analyzed how these vary when changing the values of three environment-related parameters, one agents-related parameter and one procedure. Results of the analysis show a negative dependence of all outputs on an increase to the perceived tram speed making agents think of tram being as fast as other surface means. This parameter, however, turned out to be the less influential among the inputs considered, which suggests that the analyst can safely think of setting it to its default value without further investigation and data collection. A different case is that of $r\_foot$, the parameter controlling the number of fictitious connections to add between those stops that are close to each other but not connected through any vehicle. Indeed, it turns out that not only this parameter is the most influential (the one with highest total finite change effect) with respect to all outputs, but also that, by halving its value (from a default of 200m to 100m), the network becomes less efficient. Moreover, there is evidence for significant interaction effects of this parameter when it is varied together with $p\_long$, the probability that an agent performs a "long" activity. This suggests that it would be worth to conduct more detailed data collection to gain evidence of this parameter in real life. That is, one may want to understand how far individuals are willing to walk in order to catch a vehicle which is not available at the closest stop to their location, in order to make the model more accurate in reflecting reality. 
The current version of the model clearly presents some limitations. First, as it happens for many Agent-based models, it was not possible to perform parameters' calibration, due to the lack of data about individual trips. More detailed data would allow performing a more specific sensitivity analysis, on those parameters for which the uncertainty remains high even after calibration. Another major limitation is that, when creating the edges for surface means of transport and walking paths, the existing routes have not been considered. This could be easily fixed by integrating the data from ATM \cite{site1, site2, site3, site4, site5, site6, site7, site8}, which have been used to build the graph, with additional data from OpenStreetMap \cite{site9}. Moreover, right now the model only considers the public transportation infrastructure, neglecting the other options that daily travelers have at their disposal. Hence, it could be extended to account for additional means of transport, such as private and shared surface vehicles, and to take into consideration the share of the demand for transports attributable to tourists and non-residents, for which demographic was not directly available. One last interesting development, which would benefit policy-makers and public authorities in the decision-making and planning process, would be to build, out of the model outputs, tailored metrics of system efficiency, perhaps including also sustainability and accessibility indicators.
Even with the aforementioned limitations, the proposed model presents notable advantages. It is really flexible, easily replicable and extendable to many types of scenarios. Even if it is constructed on demographic data which are not directly linked with transportation habits, the patterns resulting from the simulations are realistic replicas of what can happen in reality. Moreover, in the absence of individual travel data, the population synthesis technique adopted can be thought of as a method for generating, from demographics, some kind of OD matrices enriched with traveling times. Finally, the model can also be applied to different systems, provided that data about demographics of the population and about the infrastructure are available. 

