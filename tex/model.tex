\section{The Model}
The proposed model aims at simulating the mobility of individuals in Milan that use the public transport services, exploring the consequences of shocks and interventions that can affect directly the transport network. Specifically, the elements that constitute the model are:
\begin{itemize}
    \item The environment: the public transport network of Milan, including metro, tram, bus and filobus lines. 
    \item The agents: synthetic individuals, constituting the flow of passengers on ATM vehicles.
    \item The time: the simulation is based on data regarding an ordinary week day, from 5:00 a.m. to 1.00 a.m. of the next day and each step represents 1 minute (for a total of 1200 steps).
    \item The rules: agents, according to their schedules, move on the network, respecting the timetables and the availability in terms of capacity on the vehicles.
\end{itemize}
The idea is to generate synthetic agents, place them on the graph representing the transport network of the city of Milan, and let them move around the city according to their schedules. Individuals interact with each others as they compete to get on the same vehicle, given that vehicles have limited capacity. Moreover, they interact with the environment, since they choose the shortest path to their destinations, according to network-specific attributes (i.e. distance between stops, speeds of the vehicles, lines' timetables). 

\subsection{Data and manipulation}\label{sec3}
The model is based on different datasets describing the Milan's transport network and population. We distinguish between those needed to build the transport network and those used in the process of synthesizing the population and highlight the main steps of the manipulation process.\\ \\
The datasets used to build the network are the following:
\begin{enumerate}
    \item two datasets \cite{site1, site5} from Comune di Milano's open data, containing the main features characterizing each route of the underground lines and  of the surface transit lines respectively: an id number, the mean of transport, the line, the terminal points and their geographical coordinates, the length and the number of stops. Multiple routes can correspond to the same line. 
    \item two datasets \cite{site2, site6} from Comune di Milano's open data, containing the AMAT\footnote{Agenzia Mobilità Ambiente e Territorio, an agency, present in the municipality of Milan, that provides services to support the municipal functions in the fields of planning, programming, management, monitoring and control relating to the development of the territory and greenery, urban planning, mobility and transport public \cite{site21}.} id, the line name and number, the geographical coordinates and the location of all the stops of the metro lines and of the surface transit lines respectively.
    \item two datasets \cite{site3, site7} from Comune di Milano's open data, containing the sequence of stops for each route of the underground lines and of the surface transit lines respectively;
    \item two datasets \cite{site4, site8} from Comune di Milano's open data, containing the starting time, the ending time, the type of day (weekday, holiday, weekend) for each route of the underground lines and  of the surface transit lines respectively. 
    \item datasets \cite{site12} of GTFS\footnote{General Transit Feed Specification, common format for public transportation schedules and associated geographic information} containing, for each stop, the transit times of all the means of transport. 
    \item dataset mapping each stop, defined as a couple of geographic coordinates, to the NIL it belongs to, obtained from an open map of Milan \cite{site22}.
    \item dataset of OpenStreetMaps containing information about the geographic area of Milan and the main points of interests.
\end{enumerate} 
We discard the routes running only on weekends and holidays and, for each route, we obtain the couples of ordered stops representing the sequence. Moreover, we obtain, from the OpenStreetMaps file, the complete list of the points of interest of selected types, which we categorize into 'amenity', 'shop', 'school', 'leisure', 'office'. \\\\
The datasets used to synthesize the population are the following:
\begin{enumerate}
\setcounter{enumi}{7}
    \item dataset \cite{site18} from Sistema Statistico Integrato of Milan, containing the distribution of resident people in Milan in 2021 by age and by NIL\footnote{Nuclei d'Identità Locale, partition of the territory of Milan, introduced in the PGT (Piano di Governo del Territorio).}.
    \item dataset \cite{site10} from ISTAT, containing the distribution of oriented movements ("trasporti finalizzati") for a specific activity by class age during an ordinary weekday (updated to 2013). The population is split in 4 age classes (15-44, 25-44, 45-64 and over 65 years-old) and there are 7 possible activities.
    \item dataset \cite{site11} from ISTAT, showing the average percentage of people performing specific activities by age class in each specific 10-minute time slot during an ordinary weekday (updated to 2013). There are 6 possible activities, included that of oriented movements.
\end{enumerate}
First, we adapt the different distributions, as to have equal age classes. We discard the "volunteering" activity, for which we only have the distribution of oriented movements but we have no information about the proportion of individuals performing it, and we standardize the distributions accordingly. The final list of activities we consider is the following: sleep/eat/personal care, education, family work, paid work, free time. Finally, we rescale the activities proportions time series by the proportion of people traveling 40 minutes earlier, 40 being the average duration of daily public transit usage in Milan \cite{bib2}. The intuition is that we want the probability of each age class carrying out a specific activity in a given time slot to be higher when a greater proportion of individuals of that given class was traveling 40 minutes earlier.

\subsection{The Network}\label{sec3}

Data sources 1 to 4 are combined to construct a directed multigraph representing the public transport network of Milan: each node identifies a stop, while an edge between two stops denote the existence of a route of any transportation mean connceting them. In this way, two consecutive stops on the same transport line are joined by an edge. All the edges point in a single direction, and between two nodes there may be more than one link of different type. In our specific case, this happens when there are multiple routes that pass through two consecutive stops. We consider only the routes that are active during weekdays.

To compute the distances and better visualize the network, all the stops are placed in space using their real geographic coordinates \cite{site2,site6}.  
We add to the network edges representing a walking path connecting those stops that are not connected through transportation means but are close enough to be reached on foot from one another. In the original verion, we say that two stops are at walking distance if the geodesic distance between them is of maximum 200 meters. 
The final graph, including transport means and walking paths, contains a total of 4753 nodes and 18470 edges. 

Both nodes and edges have been endowed with some attributes.
Each node is characterised by a unique id, which corresponds to the official AMAT id. and a dictionary that contains the number of points of interests which are at most 500 meters away from the stop. For each stop, using data from dataset 7, we compute the number of reachable points of interests corresponding to schools, offices, amenities, and food courts. We say that a point of interest is reachable if its geodesic distance from the stop is less than or equal to a number r, which in the original version was set to correspond to 500 meters.  
 
Each edge is identified by the id of the two nodes that it connects and the specific line of the mean of transport (eg. 11497, 11500, TRAM12). Its attributes are: transport mode, total capacity, weight, next edges, waiting list and passengers list. The total capacity \cite{site13,site14,site15,site16} and the weight depend on the transport mode. Specifically, the weight represents the traveling time between two stops, which is computed dividing the distance between stops by the average commercial speed of the vehicle \cite{site17}. Next edges is a list which, in most cases, refers to the following edge on the same line. However, there are cases in which forks of the line require a single edge to have two “next edges”. Instead, foot edges do not have “next edges”. Finally, the waiting list and the passengers list are created as empty lists, which will be fill during the simulation to allow to identify the travellers that are waiting for and boarding each vehicle. 


 \subsection{The Agents}\label{sec3}
 
Agents in the model belong to one among 4 possible age classes: 15-24, 25-44, 45-64 and over 65. They start their daily routine at home, perform 1 to 5 activities (among education, work, leisure, self-care/eating \footnote{ISTAT actually provides a category that includes eat, sleep and other self-care. However, given the time horizon we consider (5:00 a.m  to 1:00 a.m), we assume that “sleep” can be neglected.} and housework) and, eventually, go back home. Datasets from 8 to 10 are used to create a population which resembles the real population of residents in Milan.  
The creation of the synthetic population proceeds through different steps.
%  as summarised in the algorithm \ref{alg1} below.

First of all, for each agent we draw an age class from the distribution of inhabitants of Milan. Then, we draw a NIL for the agent's house from the distribution of the NILs conditional on the chosen age group \cite{site18}. After this, we draw uniformly at random, from the stops in the selected NIL, a precise node which represents the agent’s home. 

At this point, two possible strategies can be adopted to assign a travel diary to the agents:
\begin{itemize}
    \item \textbf{Strategy 1.} For each of the five possible activities we decide whether the agent will or not perform it during the day, by drawing from a Bernoulli with parameter equal to the probability that the agent carries it out such activity in an ordinary weekday, conditional on the agent's age class. Data used to construct this probability distribution come from dataset 9. Adopting this strategy, each agent can have a variable number of activities to carry out, ranging from 1 to 5.
    \item \textbf{Strategy 2.} Agents are forced to perform a number k of activities, drawn with replacement from the distribution of the combinations of activities conditional on the age group they belong to. Specifically, we consider the choices of activities as independent from each other, so that the conditional probability of a combination of activities corresponds to the product of the conditional probabilities of the single activities.
\end{itemize}
In the original version of the model, strategy 1 has been adopted. We have realized that, on average, using stratey 1, around $20\%$ of the initially generated agents do not have any planned activity in their schedule and, therefore, we have decided to discard them. Experiments show that, on average, agents' schedule only contain 2 activities. To make the results of the two strategies comparable, the parameter k for the second strategy has been set to 2.

We assign a location to each activity selected for the agent (i.e., the specific node in the network) with probability proportional to the number of facilities corresponding to that task located nearby each node \cite{site9}. Among the possible categories proposed by OpenStreetMap, we only take into consideration the ones that match the ISTAT activities (eg. schools, universities, etc. for education; offices, co-working place, etc. for work; cinema, gyms, etc. for leisure time; restaurants, bars, etc. for eat/sleep). When the activity “housework” is selected, we assign as destination the agent’s house node. We then assign a departure time for each scheduled activity by drawing a 10-minutes time slot from the distribution of oriented movements over time conditional on the specific activity \cite{site11}, which has to be performed approximately 40 minutes after the departure (40 minutes being the average traveling time). To allow for minute-by-minute departure times, we also include a randomization term that adds or removes at most 5 minutes from the picked time. 

Finally, if the last destination of the agent is not “home”, we compute a critical time at which we let the agent depart again to go back home (time of the last activity + 40 min + average duration of last activity \cite{site11} + noise in the [-10 min, +10 min] interval). We consider two possible sets of average durations for the activities, set 2 having the values associated to education and paid work as the half of those in set 1. We assume that, if this critical time is after 10:30 p.m. the agent prefers not to use public transports to go back home, as the timetables are not favourable later in the evening.

\begin{algorithm}
\caption{Agents' generation following Strategy 1}\label{alg1}
\begin{algorithmic}
\Require $n \geq 0$
\For {each agent $i = 1, ..., n$}:
\State \textit{age\_class} $\sim Multinomial(p_l)$ \Comment{$p_l$ = P(age class l) for $l = 1, ..., 4$}
\State $nil \sim Multinomial(q_{jl})$ \Comment{$q_j$ = P(nil j $\vert$ age\_class) for $j = 1, ..., 89$}
\State $home \sim Uniform(k)$ \Comment{$ k = \frac{1}{\vert \{s: s \in nil \} \vert }$}
\For {each activity a = 1, ..., 5}:
\State $ u \sim Bernoulli(f_a)$ \Comment{$f_a$ = P(a $\vert$ age\_class)}
\If{$u = 1$}:
    \State $destination \sim Multinomial(g_s)$ \\\Comment{$g_s = \frac{\text{\# amenities of type a close to node s}}{\text{total \# of amenities of type a}}$}
    \State \textit{departure\_time} $\sim Multinomial(r_t)$ \\\Comment{$r_t$ = P(travel time $\vert$ a) + U(-5, 5)}
\EndIf
\EndFor
\EndFor
\end{algorithmic}
\end{algorithm}

At this point, for each agent we have: a unique id, the age group, the starting point (node on the network) and a schedule of activities (from 1 to 5 activities, with the related destination and the time at which the agent has to start the journey to get there), as shown in the table \ref{tab1} below. 

\begin{table}[h]
\centering
\caption{Example of agent's generation}\label{tab1}%
\begin{tabular}{ll|lll}
\multicolumn{2}{c|}{The agent} & \multicolumn{3}{c}{The schedule} \\ 
\toprule
Features      & Data      & Activity & Destination & Departure time \\ 
\midrule
Unique id & 7548 & 1) Education & V.le Bligny, 19 & 7:30\\
Age & 15-24  &  2) Work	& Duomo M3 & 14:15\\
Starting point & Corvetto M3 & 3) Leisure & Moscova & 19:25\\
\bottomrule
\end{tabular}
\end{table}

The final step to complete the synthesis of the population is to assign to each agent the list of edges that form the shortest path to reach their destinations (Example in table \ref{tab2} below). Specifically, the shortest path is calculated through the Dijkstra algorithm, using as weight the traveling time of each edge. However, not all nodes are taken into account in this process. In fact, the shortest path is computed on the subgraph of the edges constituting the lines that are active in a limited time window around the departure time. In this way, the agent can only select routes that are available at the time of their departure. Lastly, since we are working on a multigraph, it may be the case that two stops are linked by two different means of transport. In this case, the agent includes in the shortest path the edges that minimize the number of changes of transport method needed.